\documentclass[xcolor=dvipsnames,compress,12pt]{beamer}

%% General document %%%%%%%%%%%%%%%%%%%%%%%%%%%%%%%%%%
\usepackage{graphicx}
\usepackage{upgreek}
\usepackage{bm}
\usepackage{natbib}
\usepackage{xspace}
\usepackage{appendix}
%\usepackage[T1]{fontenc}
%\usepackage[scaled]{helvet}
%\renewcommand*\familydefault{\sfdefault} %% Only if the base font of the document is to be sans serif

\usepackage[LGR,T1]{fontenc} %% LGR encoding is needed for loading the package gfsneohellenic
%\usepackage[default]{gfsneohellenic}
%%%%%%%%%%%%%%%%%%%%%%%%%%%%%%%%%%%%%%%%%%%%%%%%%%%%%%

\definecolor{mycol1}{RGB}{2,0,145}
\definecolor{mycol2}{RGB}{25,23,191}
%% Beamer Layout %%%%%%%%%%%%%%%%%%%%%%%%%%%%%%%%%%
%\useoutertheme[footline=authorinstitute,subsection=false]{miniframes}
%\useoutertheme[subsection=false,shadow]{miniframes}
\useinnertheme{circles}

\useoutertheme{default}
\renewcommand{\Pr}{\mathbb{P}}

%%%%%%%%%%%%%%%%%%% Xtra footline
\makeatletter
% add a macro that saves its argument
\newcommand{\footlineextra}[1]{\gdef\insertfootlineextra{#1}}
\newbox\footlineextrabox
% add a beamer template that sets the saved argument in a box.
% The * means that the beamer font and color "footline extra" are automatically added. 
\defbeamertemplate*{footline extra}{default}{
    \begin{beamercolorbox}[ht=2.25ex,dp=1ex,leftskip=3mm,rightskip=3mm]{footline extra}
    \insertfootlineextra
    %\par\vspace{2.5pt}
    \end{beamercolorbox}
}
\addtobeamertemplate{footline}{%
    % set the box with the extra footline material but make it add no vertical space
    \setbox\footlineextrabox=\vbox{\usebeamertemplate*{footline extra}}
    \vskip -\ht\footlineextrabox
    \vskip -\dp\footlineextrabox
    \box\footlineextrabox%
}
{}
% patch \begin{frame} to reset the footline extra material
\let\beamer@original@frame=\frame
\def\frame{\gdef\insertfootlineextra{}\beamer@original@frame}
\footlineextra{}
\makeatother
\setbeamercolor{footline extra}{fg=structure.fg}% for instance
%%%%%%%%%%%%%%%%%%%%


\usefonttheme{serif}
\setbeamertemplate{navigation symbols} {}
%\setbeamerfont{title like}{shape=\textsc}
%\setbeamerfont{frametitle}{shape=\textsc}
\setbeamercolor{title}{fg= mycol2}
\setbeamercolor{subtitle}{fg=black}

\setbeamerfont{frametitle}{size=\LARGE}
\setbeamercolor{frametitle}{fg= mycol2}

\setbeamercolor*{lower separation line head}{bg= mycol1} 
\setbeamercolor*{normal text}{fg=black,bg=white} 
\setbeamercolor*{alerted text}{fg=red} 
\setbeamercolor*{example text}{fg=black} 
\setbeamercolor*{structure}{fg=black} 
 
\setbeamercolor*{palette tertiary}{fg=black,bg=black!10} 
\setbeamercolor*{palette quaternary}{fg=black,bg=black!10} 

\DeclareMathAlphabet{\msfsl}{T1}{cmr}{m}{it}
\DeclareMathAlphabet{\msyf}{OML}{cmm}{m}{it}
\newcommand{\alf}{\upalpha}
\definecolor{dgray}{gray}{0.4}
\definecolor{tan}{rgb}{0.5,0.5,0.3}

%\renewcommand{\(}{\begin{columns}}
%\renewcommand{\)}{\end{columns}}
%\newcommand{\<}[1]{\begin{column}{#1}}
%\renewcommand{\>}{\end{column}}
%%%%%%%%%%%%%%%%%%%%%%%%%%%%%%%%%%%%%%%%%%%%%%%%%% 

\title{\LARGE{Software Carpentry \\{\large \texttt{git}, testing, pipelines}}}
\date{\large Aug 23, 2013}
\author[Holder]{Mark T. Holder}
\institute[pt2012]{\inst{}Department of Ecology and Evolutionary Biology, University of Kansas}
\newcommand{\myfootnote}[1]{\let\thefootnote\relax\footnotetext{#1}}

\usepackage{pgf}
\include{positioning}
\usepackage{tikz} 
\usetikzlibrary{trees,arrows,positioning,automata}

\tikzset{suitestate/.style={circle,rounded corners,draw=blue!90,fill=blue!05,ultra thick}}
\tikzset{state/.style={rectangle,rounded corners,draw=blue!90,fill=blue!05,ultra thick}}
\tikzset{ostate/.style={rectangle,rounded corners,draw=blue!90,fill=blue!05,ultra thick}}
\tikzset{empty/.style={rectangle,rounded corners,draw=white,fill=white,thick}}
%\usepackage{paralist}
\usepackage{cancel}

\begin{document}



%%%%%%%%%%%%%%%%%%%%%%%%%%%%%%%%%%%%%%
\maketitle
%\begin{frame}
%\titlepage
%\end{frame}

%%%%%%%%%%%%%%%%%%%%%%%%%%%%%%%%%%%%%%
\begin{frame}{Motivating example}
\begin{tabular}{lcccccccccccc}
human\hskip 2mm  & C & G  & A & C & C & {A} & G & G & T & A & T &G\\
chimp\hskip 2mm  & C & G  & A & C & C & {A} & G & G & T & A & C &G\\
gorilla\hskip 2mm& C & C  & G & T & C & {C} & G & G & T & C & T &G\\
orang\hskip 2mm  & C & C  & G & C & C & {T} & G & G & T & C & C & G\\
\end{tabular}

\end{frame}

\begin{frame}{Motivating example}
\begin{tabular}{lcccccccccccc}
human\hskip 2mm  & C & {\color{red}G}  & {\color{red}A} & C & C & {A} & G & G & T & {\color{red}A} & {\color{blue}T} &G\\
chimp\hskip 2mm  & C & {\color{red}G}  & {\color{red}A} & C & C & {A} & G & G & T & {\color{red}A} & {\color{blue}C} &G\\
gorilla\hskip 2mm& C & {\color{red}C}  & {\color{red}G} & T & C & {C} & G & G & T & {\color{red}C} & {\color{blue}T} &G\\
orang\hskip 2mm  & C & {\color{red}C}  & {\color{red}G} & C & C & {T} & G & G & T & {\color{red}C} & {\color{blue}C} & G\\
\end{tabular}
\vskip 1em
{\color{red} human-with-chimp: 3}\\
{\color{blue} human-with-gorilla: 1}\\
{\color{green} human-with-orang: 0}\\
\end{frame}

%%%%%%%%%%%%%%%%%%%%%%%%%%%%%%%%%%%%%%
\begin{frame}{Motivating example}
\begin{tabular}{lcccccccccccc}
human\hskip 2mm  & C & {\color{red}G}  & {\color{red}A} & C & C & {A} & G & G & T & {\color{red}A} & {\color{blue}T} &G\\
chimp\hskip 2mm  & C & {\color{red}G}  & {\color{red}A} & C & C & {A} & G & G & T & {\color{red}A} & {\color{blue}C} &G\\
gorilla\hskip 2mm& C & {\color{red}C}  & {\color{red}G} & T & C & {C} & G & G & T & {\color{red}C} & {\color{blue}T} &G\\
orang\hskip 2mm  & C & {\color{red}C}  & {\color{red}G} & C & C & {T} & G & G & T & {\color{red}C} & {\color{blue}C} & G\\
\end{tabular}
\vskip 1em
{\color{red} human-with-chimp: 3}\\
{\color{blue} human-with-gorilla: 1}\\
{\color{green} human-with-orang: 0}\\
\vskip 1em
``best'' = largest count = 3\\
\end{frame}

%%%%%%%%%%%%%%%%%%%%%%%%%%%%%%%%%%%%%%
\begin{frame}{Motivating example}
\begin{tabular}{lcccccccccccc}
human\hskip 2mm  & C & {\color{red}G}  & {\color{red}A} & C & C & {A} & G & G & T & {\color{red}A} & {\color{blue}T} &G\\
chimp\hskip 2mm  & C & {\color{red}G}  & {\color{red}A} & C & C & {A} & G & G & T & {\color{red}A} & {\color{blue}C} &G\\
gorilla\hskip 2mm& C & {\color{red}C}  & {\color{red}G} & T & C & {C} & G & G & T & {\color{red}C} & {\color{blue}T} &G\\
orang\hskip 2mm  & C & {\color{red}C}  & {\color{red}G} & C & C & {T} & G & G & T & {\color{red}C} & {\color{blue}C} & G\\
\end{tabular}
\vskip 1em
{\color{red} human-with-chimp: 3}\\
{\color{blue} human-with-gorilla: 1}\\
{\color{green} human-with-orang: 0}\\
\vskip 1em
``best'' = largest count = 3\\
\vskip 1em
``support'' = best - second best = 3 - 1 = 2
\end{frame}

%%%%%%%%%%%%%%%%%%%%%%%%%%%%%%%%%%%%%%
\begin{frame}{Motivating example}
\begin{tabular}{lcccccccccccc}
human\hskip 2mm  & C & {\color{red}G}  & {\color{red}A} & C & C & {A} & G & G & T & {\color{red}A} & {\color{blue}T} &G\\
chimp\hskip 2mm  & C & {\color{red}G}  & {\color{red}A} & C & C & {A} & G & G & T & {\color{red}A} & {\color{blue}C} &G\\
gorilla\hskip 2mm& C & {\color{red}C}  & {\color{red}G} & T & C & {C} & G & G & T & {\color{red}C} & {\color{blue}T} &G\\
orang\hskip 2mm  & C & {\color{red}C}  & {\color{red}G} & C & C & {T} & G & G & T & {\color{red}C} & {\color{blue}C} & G\\
\end{tabular}
\vskip 1em
{\color{red} human-with-chimp: 3}\\
{\color{blue} human-with-gorilla: 1}\\
{\color{green} human-with-orang: 0}\\
\vskip 1em
``best'' = largest count = 3\\
\vskip 1em
``support'' = best - second best = 3 - 1 = 2
\end{frame}

%%%%%%%%%%%%%%%%%%%%%%%%%%%%%%%%%%%%%%
\begin{frame}{Motivating example}
Input: A file with many character matrices, each with 4 species.

\vskip 1em

Operations: 
\begin{enumerate}
    \item Calculate the support for each matrix.
    \item Report the smallest integer that is a larger than at least 95\% of the support values that you calculated in step 1.
\end{enumerate}

\end{frame}

%%%%%%%%%%%%%%%%%%%%%%%%%%%%%%%%%%%%%%
\begin{frame}{IPython is great, but...}
\Large
Storing code in files is better for:
\begin{itemize}
    \item Sharing code,
    \item Running on cluster,
    \item Reuse of code$\ldots$
\end{itemize}
\end{frame}


%%%%%%%%%%%%%%%%%%%%%%%%%%%%%%%%%%%%%%
\begin{frame}{The programmer's workflow:}
\Large
\begin{enumerate}
    \item Write code in a text editor
    \item Save the file -- save to the filesystem
    \item Execute the code from your terminal
    \item Rewrite in text editor
    \item Go back to step 2
\end{enumerate}
\end{frame}

%%%%%%%%%%%%%%%%%%%%%%%%%%%%%%%%%%%%%%
\begin{frame}{Some useful tricks}

{\large
Command-tab to switch between editor and terminal.\\
Arrow-up-key to recall previous commands. \par
}
\vskip 1em
Opening a GUI file browser in your terminal's working directory:\par
Mac: {\tt \$ open .}\par
GitBash on Windows: {\tt \$ explorer .}\par
Ubuntu: {\tt \$ nautilus .}\par

\end{frame}

%%%%%%%%%%%%%%%%%%%%%%%%%%%%%%%%%%%%%%
\begin{frame}{Version control}
\Large
Except for trivial tasks, programming involves
\begin{itemize}
    \item writing code,
    \item testing whether it works
    \item repeat
\end{itemize}

You will have many {\em versions} of your code.

A version control system helps you manage changes to a code-base
\end{frame}

 %%%%%%%%%%%%%%%%%%%%%%%%%%%%%%%%%%%%%%
\begin{frame}{{\tt git} -- a distributed version control system}
``Distributed'' here means every developer has the entire version history. There is no central server.

{\tt git} lets you
\begin{itemize}
    \item return to a previous snapshot of your code,
    \item merge work from several developers,
    \item easily share your code,
    \item easily back up your code.
\end{itemize}
\end{frame}

\begin{frame}{github $\neq$ {\tt git}}
\Large
\url{https://github.com/} is a website that offers a lot of very nice services centered around code stored in {\tt git}. But {\tt git} is a command line tool that was developed years before github.

If you are going to use {\tt git}, you probably want to create a github account to see what the fuss is about.
\end{frame}



%%%%%%%%%%%%%%%%%%%%%%%%%%%%%%%%%%%%%%
\begin{frame}{}
\Large
I know a lot of programmers.\par

I do not know {\em any} programmers who
\begin{itemize}
    \item know how to use a version control system, and
    \item prefer to write code without using version control system
\end{itemize}
\end{frame}

\begin{frame}{}
\Large

I do not know {\em any} programmers who:
\begin{itemize}
    \item know how to use a {\em distributed} version control system, and
    \item prefer centralized version control to {\em distributed} version control
\end{itemize}
\end{frame}

\begin{frame}{}
\Large
{\tt git} is clearly one of the best distributed version control systems.
\vskip 1em
If you are going to do more than dabble with programming, you should learn {\tt git}.
\vskip 1em
Learning {\tt git} will hurt a bit. It is not an easy tool.

\end{frame}

\begin{frame}{Goals for this {\tt git} Intro}
\Large
\begin{itemize}
    \item become comfortable with the ``core'' git commands
    \item gain a sense of the other important features.
\end{itemize}

\end{frame}

\begin{frame}{Core {\tt git} commands}
\large
\begin{itemize}
    \item {\tt git init}  -- create and empty repository
    \item {\tt git clone}  -- obtain an existing repository
    \item {\tt git add}  and {\tt git commit} to add content to the repository
    \item {\tt git checkout} -- copy code out of the repository to your filesystem
    \item {\tt git status} -- create a report of how {\tt git} sees the world
\end{itemize}

\end{frame}

\begin{frame}{More {\tt git} that I'll show you, but we won't practice (much)}
\large
\begin{itemize}
    \item {\tt git pull} -- grab the latest changes from a repository on another computer. Get other people's changes.
    \item {\tt git push} -- publish your latest changes to a repository on another computer. Give other people your code.
    \item {\tt git branch}
    \item {\tt git merge}
    \item {\tt git fetch}  
\end{itemize}

\end{frame}


\end{document}

